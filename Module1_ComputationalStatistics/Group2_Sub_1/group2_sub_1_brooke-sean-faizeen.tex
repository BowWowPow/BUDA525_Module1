% Options for packages loaded elsewhere
\PassOptionsToPackage{unicode}{hyperref}
\PassOptionsToPackage{hyphens}{url}
%
\documentclass[
]{article}
\usepackage{amsmath,amssymb}
\usepackage{iftex}
\ifPDFTeX
  \usepackage[T1]{fontenc}
  \usepackage[utf8]{inputenc}
  \usepackage{textcomp} % provide euro and other symbols
\else % if luatex or xetex
  \usepackage{unicode-math} % this also loads fontspec
  \defaultfontfeatures{Scale=MatchLowercase}
  \defaultfontfeatures[\rmfamily]{Ligatures=TeX,Scale=1}
\fi
\usepackage{lmodern}
\ifPDFTeX\else
  % xetex/luatex font selection
\fi
% Use upquote if available, for straight quotes in verbatim environments
\IfFileExists{upquote.sty}{\usepackage{upquote}}{}
\IfFileExists{microtype.sty}{% use microtype if available
  \usepackage[]{microtype}
  \UseMicrotypeSet[protrusion]{basicmath} % disable protrusion for tt fonts
}{}
\makeatletter
\@ifundefined{KOMAClassName}{% if non-KOMA class
  \IfFileExists{parskip.sty}{%
    \usepackage{parskip}
  }{% else
    \setlength{\parindent}{0pt}
    \setlength{\parskip}{6pt plus 2pt minus 1pt}}
}{% if KOMA class
  \KOMAoptions{parskip=half}}
\makeatother
\usepackage{xcolor}
\usepackage[margin=1in]{geometry}
\usepackage{graphicx}
\makeatletter
\newsavebox\pandoc@box
\newcommand*\pandocbounded[1]{% scales image to fit in text height/width
  \sbox\pandoc@box{#1}%
  \Gscale@div\@tempa{\textheight}{\dimexpr\ht\pandoc@box+\dp\pandoc@box\relax}%
  \Gscale@div\@tempb{\linewidth}{\wd\pandoc@box}%
  \ifdim\@tempb\p@<\@tempa\p@\let\@tempa\@tempb\fi% select the smaller of both
  \ifdim\@tempa\p@<\p@\scalebox{\@tempa}{\usebox\pandoc@box}%
  \else\usebox{\pandoc@box}%
  \fi%
}
% Set default figure placement to htbp
\def\fps@figure{htbp}
\makeatother
\setlength{\emergencystretch}{3em} % prevent overfull lines
\providecommand{\tightlist}{%
  \setlength{\itemsep}{0pt}\setlength{\parskip}{0pt}}
\setcounter{secnumdepth}{-\maxdimen} % remove section numbering
\usepackage{bookmark}
\IfFileExists{xurl.sty}{\usepackage{xurl}}{} % add URL line breaks if available
\urlstyle{same}
\hypersetup{
  pdftitle={Group2\_HW1\_Submision\_Brooke-Sean-Faizaan},
  hidelinks,
  pdfcreator={LaTeX via pandoc}}

\title{Group2\_HW1\_Submision\_Brooke-Sean-Faizaan}
\author{}
\date{\vspace{-2.5em}2025-08-29}

\begin{document}
\maketitle

install.packages(``ISLR'') library(ISLR) dim(College) \# rows, cols
help(College)

vStore \textless- matrix(0,18,1000)

for(i in 1:10000) \{ vStore{[},i{]}=sample(Wage\$wage,1000,replace=TRUE)
\}

\begin{center}\rule{0.5\linewidth}{0.5pt}\end{center}

install.packages(``ISLR'') library(ISLR) dim(College) head(College) \#
storeIntVar \textless- table\$GetDataInTable

apps\textless-College\$Apps \# we run sampling distros based on loads of
different seeds. \# Each time we random this, we will get different
sampling data. \# working hard to then run the code multiple times to
see the difference in \# variation \# -- This is the old way --
set.seed(1900)

\section{setting up an empty var for use later.
Delegates}\label{setting-up-an-empty-var-for-use-later.-delegates}

bootsApp\textless-NULL \# First thing that looks like code. thank you.
\# ``most efficient way'' for(j in 1:10000)\{
bootsApp\textless-c(bootsApp,median(sample(apps,length(apps),replace =
TRUE))) \} bootsApp\textless-NULL \# writing as dplyr library(dplyr)
for(j in 1:1000)\{ \# samples apps as size apps - with replacement. Then
take that and calc the \# medium of that example \# said about changing
medium to the other one and it ``would work'' \# also we need to change
the seed
my\_samp=apps\%\textgreater\%sample(.,length(.),replace=TRUE)\%\textgreater\%median()
bootsApp\textless-c(bootsApp,my\_samp) \}

hist(bootsApp)

quantile(bootsApp,c(.1,.9))

\end{document}
